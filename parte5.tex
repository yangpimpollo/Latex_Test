\begin{enumerate}
\setcounter{enumi}{62}
    \item Demuestre que si $X\thicksim t$, entonces $X$ tiene una distribución de Cauchy.
    \\
    \textbf{Solución:}\\
    $~~~~~~~~~Teorema:~$La distribución estándar de Cauchy es un caso especial de la distribución t de Student cuando n = 1; donde n es grados de libertad. \\
    $~~~~~~~~~Prueba:~$La distribución t de Student tiene una función de densidad de probabilidad siguiente
    $$f(x)=\frac{\Gamma(\frac{n+1}{2})}{\sqrt{n\pi}\Gamma(\frac{n}{2})(\frac{x^2}{n}+1)^{(n+1)/2)}}~~~~~~~~~-\infty<x<\infty
    $$
    Cuando $n=1$, esto se convierte en:
    $$f(x)=\frac{\Gamma(1)}{\Gamma(\frac{1}{2})\sqrt{\pi}(x^2+1)}~~~~~\to~~~~~ f(x)=\frac{1}{\pi(1+x^2)}
    $$
    que es la función de densidad de probabilidad de la distribución estándar de Cauchy.
    
\end{enumerate}
\newpage